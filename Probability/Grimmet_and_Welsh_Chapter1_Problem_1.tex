\section{Grimmet and Welsh Chapter 1 Problem 1}
\subsection{Question}

A fair die is thrown $n$ times. Show that the probability that there are an even number of sixes is 
 
\begin{equation*} 
\mathbf{P}_n := \frac{1}{2}\left[1 + \left({\frac{2}{3}}\right)^n\right] 
\end{equation*}

\subsection{Some general observations}
Looking at specific values of $n$ seems a good idea to get a feel for the behaviour of $\mathbf{P}_n$ and to validate any formulae we obtain. \\
$\mathbf{P}_0 = 0$ ,
$\mathbf{P}_1 = \frac{5}{6}$ ,
$\mathbf{P}_2 = \frac{1}{36} + \frac{25}{36} = \frac{26}{36}$ \\

For very large $n$, the number of sixes will be broadly spread around the value $\frac{n}{6}$ with broadly similar amounts of even and odd values, so we should also expect  
\begin{equation*} 
\lim_{n \rightarrow \infty} \mathbf{P}_n = \frac{1}{2} 
\end{equation*}

\subsection{Answer 1}
% TAG is the printed reference after the equation, LABEL is only used for internal back-referencing - i.e.
% Please see \eqref{Formula1}  will display : Please see (F1)

A direct approach! \\

Firstly, consider when $n$ is even (only to save messy-looking equations) :

Let ${}_i\mathbf{P}_n$ be the probability of throwing $i$ sixes from $n$ throws.  \\
Using \eqref{Formula1.1} :
\begin{equation*} 
\mathbf{P}_n = {}_0\mathbf{P}_n + {}_2\mathbf{P}_n + \ldots + {}_n\mathbf{P}_n    
\end{equation*} 

But ${}_i\mathbf{P}_n = \binom{n}{i} \left(\frac{1}{6}\right)^i \left(\frac{5}{6}\right)^{n-i}  $, so
\begin{equation*} 
\mathbf{P}_n = \binom{n}{0} \left(\frac{1}{6}\right)^0 \left(\frac{5}{6}\right)^{n} + 
\binom{n}{2} \left(\frac{1}{6}\right)^2 \left(\frac{5}{6}\right)^{n-2} + \ldots + \binom{n}{n} \left(\frac{1}{6}\right)^n \left(\frac{5}{6}\right)^{0}
\end{equation*}

To evaluate this, note that : 
\begin{equation*} 
\left(1+x\right)^n + \left(1-x\right)^n = 2\left(\binom{n}{0}x^0 + \binom{n}{2}x^2 + \ldots + \binom{n}{n}x^n\right)
\end{equation*} 

Setting $x = \frac{1}{5}$ :
\begin{equation*} 
\frac{1}{2}\left[\left(\frac{6}{5}\right)^n + \left(\frac{4}{5}\right)^n \right] = 
\binom{n}{0}\left(\frac{1}{5}\right)^0 + \binom{n}{2}\left(\frac{1}{5}\right)^2 + \ldots + \binom{n}{n}\left(\frac{1}{5}\right)^n 
\end{equation*} 

Multiplying by $\left(\frac{5}{6}\right)^n$ we get :
\begin{equation*} 
\mathbf{P}_n = \left(\frac{5}{6} \right)^n \times \left(\frac{1}{2} \right) 
\left(\left(\frac{6}{5} \right)^n + \left(\frac{4}{5} \right)\right)^n  
= \frac{1}{2} \left[1 + \left({\frac{2}{3}}\right)^n\right] 
\end{equation*}
 
For when $n$ is odd, very similar logic applies. The revised equations corresponding to above are : 
\begin{equation*} 
\mathbf{P}_n = {}_0\mathbf{P}_n + {}_2\mathbf{P}_n + \ldots + {}_{n-1}\mathbf{P}_n  
\end{equation*}
\begin{equation*} 
\mathbf{P}_n = \binom{n}{0} \left(\frac{1}{6}\right)^0 \left(\frac{5}{6}\right)^{n} + 
\binom{n}{2} \left(\frac{1}{6}\right)^2 \left(\frac{5}{6}\right)^{n-2} + \ldots + \binom{n}{n-1} \left(\frac{1}{6}\right)^{n-1} \left(\frac{5}{6}\right)^{1}
\end{equation*}
\begin{equation*} 
\left(1+x\right)^n + \left(1-x\right)^n = 2\left(\binom{n}{0}x^0 + \binom{n}{2}x^2 + \ldots + \binom{n}{n-1}x^{n-1}\right) 
\end{equation*} 
and the same result is obtained.

\subsection{Answer 2}
Using a recurrence relation. \\
If we throw a die $n+1$ times then to obtain an even number of sixes, either the first throw is not a six (with probabilty $5/6$) and we need an 
even number of sixes from the remaining $n$ throws, or the first throw is a six (with probability $1/6$) 
and we need an odd number of sixes from the remaining $n$ throws.
\\
So, using the partition theorem \eqref{Formula1.9} conditioning on the first throw :

\begin{equation*} 
\mathbf{P}_{n+1} = \frac{5}{6}\mathbf{P}_{n} + \frac{1}{6}(1-\mathbf{P}_{n})
\Rightarrow \mathbf{P}_{n+1} = \frac{1}{6} + \frac{2}{3}\mathbf{P}_{n}
\end{equation*} 

Using ``standard'' methods, the auxiliary equation is $\theta - \frac{2}{3} = 0$, yielding the complementary solution : 
\begin{equation*} 
A\left(\frac{2}{3}\right)^n
\end{equation*} 
and a particular solution $\mathbf{P}_{n} = \frac{1}{2}$

So, the general solution is $\mathbf{P}_{n} = \frac{1}{2} + A\left(\frac{2}{3}\right)^n$, but as we know $P_0 = 1$, then $A = \frac{1}{2}$.

Therefore, as before :
\begin{equation*} 
\mathbf{P}_n = \frac{1}{2}\left[1 + \left({\frac{2}{3}}\right)^n\right] 
\end{equation*}

\subsection{Answer 3}
Given that we are provided with the answer, we can use induction on $n$. \\
This is easiest but least pleasing (as induction often is) as we don't really gain any insight into why it is true\ldots.

The statement is clearly true for $n = 0$ as  $\mathbf{P}_0 = 0$. 

Assume it is true for $n = k$. Therefore :
\begin{equation*} 
\mathbf{P}_k = \frac{1}{2}\left[1 + \left({\frac{2}{3}}\right)^k\right] 
\end{equation*}

But, as in Answer 2,
\begin{equation*} 
\mathbf{P}_{k+1} = \frac{1}{6} + \frac{2}{3}\mathbf{P}_{k}
= \frac{1}{6} + \frac{2}{3} \times \frac{1}{2} \times \left[1 + \left({\frac{2}{3}}\right)^k\right]
= \frac{1}{2}\left[1 + \left({\frac{2}{3}}\right)^{k+1}\right] 
\end{equation*} 
so it is also true for $n = k+1$ and so true for all $n \geq 0$.

\subsection{Final comments}
As expected, the formula does indeed tend to $\frac{1}{2}$ as $n \rightarrow \infty$. 
Also, an obvious consequence is that the probability that there are an odd number of sixes in $n$ throws is 
 
\begin{equation*} 
\mathbf{P}_n = \frac{1}{2}\left[1 - \left({\frac{2}{3}}\right)^n\right] 
\end{equation*}