\section{Grimmet and Welsh Chapter 1 Problem 12b}
\subsection{Question}

Part (a) of this question was to prove the formula \eqref{Formula1.5} : 
\begin{equation} 
\mathbf{P} (\bigcup_{1 \leq i \leq n} A_i) = \sum_{1 \leq i \leq n} \mathbf{P}(A_i) - 
\sum_{1 \leq i < j \leq n} \mathbf{P}(A_i \cap A_j) + \ldots 
+ (-1)^{n+1} \mathbf{P}(\bigcap_{1 \leq i \leq n} A_i) 
\end{equation}

which we'll assume has been done!

One evening, a bemused lodge-porter tried to hang $n$ keys on their $n$ hooks, but only managed to hang them independently and at random. 
There was no limit to the number of keys which could be hung on any hook. Otherwise, or by using (a), find an expression for the probability that at least one key was hung on its own hook.\\
The following morning the porter was rebuked by the Bursar, so that in the evening he was careful to hang only one key on each hook. 
But he still only managed to hang them independently and at random. Find an expression for the probability that no key was then hung on its own hook.\\
Find the limits of both expressions as $n$ tends to infinity.

You may assume that 
\begin{equation} 
e^x = \sum_{r=0}^\infty \frac{x^r}{r!} = \lim_{N\rightarrow \infty} \left(1 + \frac{x}{N}\right)^N
\label{Problem12_1}
\end{equation} 
for all real $x$.

\subsection{Answer to first bit}
The probability $\mathbf{P}_n$ that NO key was hung on its own hook is  
\begin{equation*} 
\mathbf{P}_n = \left(\frac{n-1}{n}\right)^n   
\end{equation*}
as each key has $n-1$ positions out of $n$ to be placed on to avoid being placed on its own hook.\\
So, the probability that at least one key was hung on its own hook is
\begin{equation*} 
1 - \mathbf{P}_n = 1 - \left(1 - \frac{1}{n}\right)^n   
\end{equation*} 

Putting $x = -1$ in \eqref{Problem12_1} we see that $\lim_{n \rightarrow \infty} 1 - \mathbf{P}_n = 1 - \frac{1}{e} $ which is approximately $0.632$.

\subsection{Answer to second bit}
Let $A_i$ be the event that key ``i'' is on hook ``i''. $(1 \leq i \leq n)$ \\
The probability $\mathbf{P}_n$ that at least one key was hung on its own hook (with no duplicates allowed) is :
\begin{equation*} 
\mathbf{P}_n = \mathbf{P}\left(\bigcup_1^n A_i\right)   
\end{equation*}
i.e. one or more of the events $A_1$, $A_2$, has occurred. \\
From part (a), this is equal to 
\begin{equation} 
\mathbf{P} (\bigcup_{1 \leq i \leq n} A_i) = \sum_{1 \leq i \leq n} \mathbf{P}(A_i) - 
\sum_{1 \leq i < j \leq n} \mathbf{P}(A_i \cap A_j) + \ldots 
+ (-1)^{n+1} \mathbf{P}(\bigcap_{1 \leq i \leq n} A_i) 
\end{equation}

Looking specifically at $\sum_{1 \leq i < j \leq n} \mathbf{P}(A_i \cap A_j)$, there are $\binom{n}{2}$ ways of choosing the values of $i$ and $j$, and for each pair of values,
key $i$ has $n$ possible hooks, and given that no multiples of keys are allowed on any hook, there are $n-1$ remaining possible hooks for key $j$. \\
So this term has value $\binom{n}{2} \frac{1}{n(n-1)}$ with similar arguments for the other terms.

\begin{IEEEeqnarray*}{rCl}
\implies \mathbf{P} (\bigcup_{1 \leq i \leq n} A_i) = & &
\\ 
\\
\binom{n}{1} \frac{1}{n} - \binom{n}{2} \frac{1}{n(n-1)} + \binom{n}{3} \frac{1}{n(n-1)(n-2)} - & \cdots 
+ (-1)^{n+1} \binom{n}{n} \frac{1}{n(n-1)(n-2)\ldots 1} & 
\\
% = \frac{1}{1!} -  \frac{1}{2!} + \frac{1}{3!} - \cdots  +  (-1)^{n+1}\frac{1}{n!} & &
\label{Problem12_2}
\end{IEEEeqnarray*}
\begin{equation} 
= \frac{1}{1!} -  \frac{1}{2!} + \frac{1}{3!} - \cdots  +  (-1)^{n+1}\frac{1}{n!}
\end{equation}

Again, putting $x = -1$ in \eqref{Problem12_1} and substracting from $1$, we see that (noting the slightly different definition of $\mathbf{P}_n$ ) : 
$\lim_{n \rightarrow \infty} \mathbf{P}_n = 1 - \frac{1}{e} $

And, as the question asked for the probability that no key was hung on its own hook, the answer(s) are just obtained by subtracting from $1$. 

\subsection{Comment}
The probabilities for at least one key on its own hook being $\sim 1 - e^{-1}$ for both scenarios might seem counter-intuitive.
If, however, we use part (a) to solve the first question with multiple keys allowed on any hook, the only difference is that the denominators in \eqref{Problem12_2} are powers of $n$ rather than $n(n-1)(n-2)\ldots$ \\
For the third term for instance, the expression will be 
\begin{equation} 
\binom{n}{3} \frac{1}{n^3} = \frac{1}{3!} \left((1-\frac{2}{n})(1-\frac{1}{n})(1-\frac{0}{n})\right) 
\end{equation}

With careful handling, the whole expression can still be shown to be $1 - \frac{1}{e} + \mathcal{O}\frac{1}{n}$ which has the same limit.
