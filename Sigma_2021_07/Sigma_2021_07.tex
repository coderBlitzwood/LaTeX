\documentclass[a4paper,10pt]{article}
\pagestyle{plain}
%\pdfpagewidth=\paperwidth
%\pdfpageheight=\paperheight
\addtolength{\textheight}{3.1cm}
\addtolength{\textwidth}{2.1cm}
\addtolength{\voffset}{-2.1cm}

\usepackage{amsmath}
\usepackage{amssymb}
\usepackage{mathrsfs}
\usepackage[retainorgcmds]{IEEEtrantools}
\usepackage{enumerate}
\usepackage[table,xcdraw]{xcolor}
\usepackage{longtable}
\usepackage{xcolor}
\author{Andy Mitchell}
% Define the title
\title{Probability problems for Sigma - part 1}
\begin{document} 
\maketitle 
\begin{flushleft} 
%\section{Introduction} 
Hello Sigma members! \\
After a long gap I have, at last, managed to find some time to revisit some mathematics and started reading an old book I have on probability. 
That book is ``Probability - an Introduction'' by Grimmet and Welsh (1985) from which I've investigated some of the problems/exercises from the first chapter which I hope you'll find interesting.\\

An Appendix to this submission contains a quick guide on recurrence relations (also based on a similar section from the same book) which seem to be useful 
in many situations when looking at probabilty problems.
Hopefully, I'll be looking at some extra problems for a subsequent newsletter.

\section{Grimmet and Welsh Chapter 1 Problem 1}
\subsection{Question}

A fair die is thrown $n$ times. Show that the probability that there are an even number of sixes is 
 
\begin{equation*} 
\mathbf{P}_n := \frac{1}{2}\left[1 + \left({\frac{2}{3}}\right)^n\right] 
\end{equation*}

\subsection{Some general observations}
Looking at specific values of $n$ seems a good idea to get a feel for the behaviour of $\mathbf{P}_n$ and to validate any formulae we obtain. \\
$\mathbf{P}_0 = 0$ ,
$\mathbf{P}_1 = \frac{5}{6}$ ,
$\mathbf{P}_2 = \frac{1}{36} + \frac{25}{36} = \frac{26}{36}$ \\

For very large $n$, the number of sixes will be broadly spread around the value $\frac{n}{6}$ with broadly similar amounts of even and odd values, so we should also expect  
\begin{equation*} 
\lim_{n \rightarrow \infty} \mathbf{P}_n = \frac{1}{2} 
\end{equation*}

\subsection{Answer 1}
% TAG is the printed reference after the equation, LABEL is only used for internal back-referencing - i.e.
% Please see \eqref{Formula1}  will display : Please see (F1)

A direct approach! \\

We will use this standard result : If $A_1, A_2, \ldots$ are disjoint, i.e. $A_i \cup A_j = \emptyset$ for $i \neq j$ : 
\begin{equation} 
%\mathrm{P} \mathit{P} \mathnormal{P} \mathcal{P} \mathscr{P} \mathfrak{P} \mathbb{P} \mathbf{P} \mathsf{P} \mathtt{P}  
\mathbf{P} (\bigcup_{i=1}^\infty A_i) = \sum_{i=1}^\infty \mathbf{P}(A_i) 
\tag{F1.1} \label{Formula1.1}
\end{equation}

Firstly, consider when $n$ is even (only to save messy-looking equations) :

Let ${}_i\mathbf{P}_n$ be the probability of throwing $i$ sixes from $n$ throws.  \\
Using \eqref{Formula1.1} :
\begin{equation*} 
\mathbf{P}_n = {}_0\mathbf{P}_n + {}_2\mathbf{P}_n + \ldots + {}_n\mathbf{P}_n    
\end{equation*} 

But ${}_i\mathbf{P}_n = \binom{n}{i} \left(\frac{1}{6}\right)^i \left(\frac{5}{6}\right)^{n-i}  $, so
\begin{equation*} 
\mathbf{P}_n = \binom{n}{0} \left(\frac{1}{6}\right)^0 \left(\frac{5}{6}\right)^{n} + 
\binom{n}{2} \left(\frac{1}{6}\right)^2 \left(\frac{5}{6}\right)^{n-2} + \ldots + \binom{n}{n} \left(\frac{1}{6}\right)^n \left(\frac{5}{6}\right)^{0}
\end{equation*}

To evaluate this, note that : 
\begin{equation*} 
\left(1+x\right)^n + \left(1-x\right)^n = 2\left(\binom{n}{0}x^0 + \binom{n}{2}x^2 + \ldots + \binom{n}{n}x^n\right)
\end{equation*} 

Setting $x = \frac{1}{5}$ :
\begin{equation*} 
\frac{1}{2}\left[\left(\frac{6}{5}\right)^n + \left(\frac{4}{5}\right)^n \right] = 
\binom{n}{0}\left(\frac{1}{5}\right)^0 + \binom{n}{2}\left(\frac{1}{5}\right)^2 + \ldots + \binom{n}{n}\left(\frac{1}{5}\right)^n 
\end{equation*} 

Multiplying by $\left(\frac{5}{6}\right)^n$ we get :
\begin{equation*} 
\mathbf{P}_n = \left(\frac{5}{6} \right)^n \times \left(\frac{1}{2} \right) 
\left(\left(\frac{6}{5} \right)^n + \left(\frac{4}{5} \right)\right)^n  
= \frac{1}{2} \left[1 + \left({\frac{2}{3}}\right)^n\right] 
\end{equation*}
 
For when $n$ is odd, very similar logic applies. The revised equations corresponding to above are : 
\begin{equation*} 
\mathbf{P}_n = {}_0\mathbf{P}_n + {}_2\mathbf{P}_n + \ldots + {}_{n-1}\mathbf{P}_n  
\end{equation*}
\begin{equation*} 
\mathbf{P}_n = \binom{n}{0} \left(\frac{1}{6}\right)^0 \left(\frac{5}{6}\right)^{n} + 
\binom{n}{2} \left(\frac{1}{6}\right)^2 \left(\frac{5}{6}\right)^{n-2} + \ldots + \binom{n}{n-1} \left(\frac{1}{6}\right)^{n-1} \left(\frac{5}{6}\right)^{1}
\end{equation*}
\begin{equation*} 
\left(1+x\right)^n + \left(1-x\right)^n = 2\left(\binom{n}{0}x^0 + \binom{n}{2}x^2 + \ldots + \binom{n}{n-1}x^{n-1}\right) 
\end{equation*} 
and the same result is obtained.

\subsection{Answer 2}
Using a recurrence relation (see the Appendix). \\
If we throw a die $n+1$ times then to obtain an even number of sixes, either the first throw is not a six (with probabilty $5/6$) and we need an 
even number of sixes from the remaining $n$ throws, or the first throw is a six (with probability $1/6$) 
and we need an odd number of sixes from the remaining $n$ throws.
\\
So, conditioning on the first throw :

\begin{equation*} 
\mathbf{P}_{n+1} = \frac{5}{6}\mathbf{P}_{n} + \frac{1}{6}(1-\mathbf{P}_{n})
\Rightarrow \mathbf{P}_{n+1} = \frac{1}{6} + \frac{2}{3}\mathbf{P}_{n}
\end{equation*} 

Using ``standard'' methods, the auxiliary equation is $\theta - \frac{2}{3} = 0$, yielding the complementary solution : 
\begin{equation*} 
A\left(\frac{2}{3}\right)^n
\end{equation*} 
and a particular solution $\mathbf{P}_{n} = \frac{1}{2}$

So, the general solution is $\mathbf{P}_{n} = \frac{1}{2} + A\left(\frac{2}{3}\right)^n$, but as we know $P_0 = 1$, then $A = \frac{1}{2}$.

Therefore, as before :
\begin{equation*} 
\mathbf{P}_n = \frac{1}{2}\left[1 + \left({\frac{2}{3}}\right)^n\right] 
\end{equation*}

\subsection{Answer 3}
Given that we are provided with the answer, we can use induction on $n$. \\
This is easiest but least pleasing (as induction often is) as we don't really gain any insight into why it is true\ldots.

The statement is clearly true for $n = 0$ as  $\mathbf{P}_0 = 0$. 

Assume it is true for $n = k$. Therefore :
\begin{equation*} 
\mathbf{P}_k = \frac{1}{2}\left[1 + \left({\frac{2}{3}}\right)^k\right] 
\end{equation*}

But, as in Answer 2,
\begin{equation*} 
\mathbf{P}_{k+1} = \frac{1}{6} + \frac{2}{3}\mathbf{P}_{k}
= \frac{1}{6} + \frac{2}{3} \times \frac{1}{2} \times \left[1 + \left({\frac{2}{3}}\right)^k\right]
= \frac{1}{2}\left[1 + \left({\frac{2}{3}}\right)^{k+1}\right] 
\end{equation*} 
so it is also true for $n = k+1$ and so true for all $n \geq 0$.

\subsection{Final comments}
As expected, the formula does indeed tend to $\frac{1}{2}$ as $n \rightarrow \infty$. 
Also, an obvious consequence is that the probability that there are an odd number of sixes in $n$ throws is 
 
\begin{equation*} 
\mathbf{P}_n = \frac{1}{2}\left[1 - \left({\frac{2}{3}}\right)^n\right] 
\end{equation*}






\section{Grimmet and Welsh Chapter 1 Exercise 27}
\subsection{Question}

A biased coin shows heads with probability $p = 1 - q$ whenever it is tossed. \\
Let $u_n$ be the probability that, in $n$ tosses, no pair of heads occur successively. Show that for $n \geq 1$
 
\begin{equation*} 
u_{n+2} = q u_{n+1} + pq u_n 
\end{equation*}

and find $u_n$ by the usual method if $p = \frac{2}{3}$

\subsection{Answer 1}
% TAG is the printed reference after the equation, LABEL is only used for internal back-referencing - i.e.
% Please see \eqref{Formula1}  will display : Please see (F1)

We will be using the ``Partition Theorem'' : \\
If $B_i \cap B_J = \emptyset$ and $\cup_i (B_i) = \Omega$ and $\mathbf{P}(B_i) > 0 \qquad \forall i$ :
\begin{equation} 
\mathbf{P}(A) = \sum_i \mathbf{P}(A \mid B_i)\mathbf{P}(B_i) \qquad \text{ for all } A
   \tag{F1.9} \label{Formula1.9}
\end{equation}

Let $A_n$ represent the event that, in $n$ tosses, no pair of heads occur successively
Let $B_i$ be the event that the first $(i-1)$ tosses are heads (``H") and the $i$th toss is tails (``T") (and $i \geq 1$) \\
So $B_1$ represents the events $T\ldots$ ,\\
$B_2$ represents the events $HT\ldots$ ,\\
$B_3$ represents the events $HHT\ldots$  etc.\\

Note that if $\Omega$ represents all possible events
\begin{equation*} 
\bigcup_i B_i = \Omega    \text{ and }   B_i \cap B_j = \emptyset     \text{ for all } i \neq j
\end{equation*} \\

So, we can use the Partition Theorem \eqref{Formula1.9} as follows.

\begin{equation*} 
u_{n+2} = \mathbf{P}(A_{n+2}) = \mathbf{P}(A_{n+2} \mid B_1)\mathbf{P}(B_1) + \mathbf{P}(A_{n+2} \mid B_2)\mathbf{P}(B_2) + \ldots 
\end{equation*}
But, $\mathbf{P}(A_{n+2} \mid B_1)$ looks at the event given a beginning ``T'', which 'resets' the starting state to having $n+1$ tosses 
in which to get no pair of heads (since the tail does not affect the situation for subsequent throws). This also applies to $\mathbf{P}(A_{n+2} \mid B_2)$. \\
$\mathbf{P}(A_{n+2} \mid B_3)$ , $\mathbf{P}(A_{n+2} \mid B_4)\ldots$ are all $0$ since the outcome of no pair of heads has failed for all of them. So,  
\begin{equation*} 
u_{n+2} = u_{n+1} q + u_n pq + 0 + 0 + \ldots 
\end{equation*}


For the second part, we substitute $p = \frac{2}{3}$ to obtain
\begin{equation*} 
u_{n+2} = \frac{1}{3} u_{n+1} + \frac{2}{9} u_n  \implies 9u_{n+2} - 3u_{n+1} - 2u_n = 0
\end{equation*}
and we also note that $u_1 = 1$, and $u_2 = \frac{5}{9}$ \\

Using ``standard'' methods (see the Appendix), the auxiliary equation is $9\theta^2 - 3\theta - 2 = 0 \rightarrow (3\theta+1)(3\theta-2) = 0$ , yielding the general solution : 
\begin{equation*} 
u_n = A\left(\frac{-1}{3}\right)^n + B\left(\frac{2}{3}\right)^n \text{ for some } A \text{ and } B.
\end{equation*} 

\begin{equation*} 
u_1 = 1 \implies -\frac{1}{3}A + \frac{2}{3}B = 1 \rightarrow 2B - A = 3
\end{equation*} 
\begin{equation*} 
u_2 = \frac{5}{9} \implies A + 4B = 5
\end{equation*} 
which leads to 
\begin{equation*} 
B = \frac{4}{3} \text{ and } A = -\frac{1}{3}
\end{equation*} 

\begin{equation} 
\therefore u_n = \frac{4}{3} \left(\frac{2}{3}\right)^n - \frac{1}{3} \left(\frac{-1}{3}\right)^n   \text{ for } n \geq 1
\label{Ex27_Result}
% TAG is the printed reference after the equation, LABEL is only used for internal back-referencing - i.e.
% Please see \eqref{Formula1}  will display : Please see (F1)
\end{equation} 

There are other approaches considered later.

\subsection{Comments}
Below is a table of probabilities for the first few values of $n$. As it is easy to make mistakes deriving formulae, it's a good idea to verify results by using a separate mechanism - in this case, a simple Java program which
 'makes', for each $n$, $n$ tosses of a coin millions of times and counts the proportion which don't contain 2 consecutive heads. \\
As you can see, the results suggest that the answers above are (probably) correct! \\ 
\begin{table}[!hbp]
\begin{tabular}{|l|l|l|l|}
\hline
\textbf{n} & \textbf{$\frac{4}{3} \left(\frac{2}{3}\right)^n - \frac{1}{3} \left(\frac{-1}{3}\right)^n$} & \textbf{Ratio} & \textbf{Java program} \\ \hline
1          & 1                &                & 1                                 \\ \hline
2          & 0.5555555556     & 0.555556       & 0.55553827                        \\ \hline
3          & 0.4074074074     & 0.733333       & 0.407404815                       \\ \hline
4          & 0.2592592593     & 0.636364       & 0.25929586                        \\ \hline
5          & 0.1769547325     & 0.68254        & 0.176948485                       \\ \hline
6          & 0.1165980796     & 0.658915       & 0.116604615                       \\ \hline
7          & 0.0781893004     & 0.670588       & 0.078175775                       \\ \hline
8          & 0.0519737845     & 0.664717       & 0.05198466                        \\ \hline
9          & 0.0346999949     & 0.667644       & 0.034695565                       \\ \hline
10         & 0.0231163949     & 0.666179       & 0.023142995                       \\ \hline
11         & 0.0154165749     & 0.666911       & 0.01540314                        \\ \hline
12         & 0.0102758349     & 0.666545       & 0.01027086                        \\ \hline
\end{tabular}
\end{table}

Also note the successvie ratio of probabilities for increasing values of $n$ which, as expected, converge to the dominant power $2/3$ in \eqref{Ex27_Result} above.   

\subsection{Alternative solution}
We can also get the same result by letting $v_n$ be the probability that, in $n$ tosses, a pair of heads does occur successively.
Defining $A_n^*$ as the event that, in $n$ tosses, a pair of heads does occur successively, and $B_i$ exactly as before, 

\begin{IEEEeqnarray*}{l}
v_{n+2} = \mathbf{P}(A_{n+2}^*) = \mathbf{P}(A_{n+2}^* \mid B_1)\mathbf{P}(B_1) + \mathbf{P}(A_{n+2}^* \mid B_2)\mathbf{P}(B_2) + \ldots \\
\qquad = v_{n+1} q + v_n pq + 1 \cdot pq + 1 \cdot p^2q + 1 \cdot p^3q + \ldots \\
\qquad = v_{n+1} q + v_n pq + \frac{p^2 q}{ 1-p } 
\quad = v_{n+1} q + v_n pq + p^2 
\end{IEEEeqnarray*}

from which we can either substitute $u_n = 1 - v_n$ to get the previous relation, or proceed to solve it for a particular value of $p$ and substitute later\ldots

Solving for $p = \frac{2}{3}$ we obtain 
\begin{equation} 
v_n = 1 - \frac{4}{3} \left(\frac{2}{3}\right)^n + \frac{1}{3} \left(\frac{-1}{3}\right)^n  \equiv 1 - u_n
% TAG is the printed reference after the equation, LABEL is only used for internal back-referencing - i.e.
% Please see \eqref{Formula1}  will display : Please see (F1)
\end{equation}






\section{Grimmet and Welsh Chapter 1 Problem 7}
\subsection{Question}

Two people toss a fair coin $n$ times. Show that the probability that they throw an equal number of heads is 
\begin{equation*} 
\binom{2n}{n}\left(\frac{1}{2}\right)^{2n}
\end{equation*} 

\subsection{Answer 1}
The hints section of the book suggests this.\\
If $X$ and $Y$ are the numbers of heads tossed by each person :

\begin{equation*} 
\mathbf{P}(X = Y) = \sum_k \mathbf{P}(X = k)\mathbf{P}(Y = k) =  \sum_k \mathbf{P}(X = k)\mathbf{P}(Y = n - k)  
\end{equation*}
The last statement works because the coin is fair, so this is essentially saying that $k$ heads has the same probability as $k$ tails, which is the same event as $n - k$ heads. 
But,
\begin{equation*} 
\sum_k \mathbf{P}(X = k)\mathbf{P}(Y = n - k) =   \mathbf{P}(X + Y = n)
\end{equation*}

which is the probability of, regardless of how they are distributed amongst each person, $n$ heads result from $2n$ tosses, which is just  $\binom{2n}{n}\left(\frac{1}{2}\right)^{2n}$

\subsection{Answer 2}
Here is a more direct (and longer!) approach.
\begin{equation*} 
\mathbf{P}(X = Y) = \sum_k \mathbf{P}(X = k)\mathbf{P}(Y = k) = 
 \sum_k \frac{1}{2^k}\frac{1}{2^{n-k}} \binom{n}{k} \frac{1}{2^k}\frac{1}{2^{n-k}} \binom{n}{k} 
  = \frac{1}{2^{2n}}\sum_{k=0}^n \binom{n}{k}^2
\end{equation*}

Now,
\begin{equation*}
(1 + x)^n = \binom{n}{0}x^0 + \binom{n}{1}x^1 + \binom{n}{2}x^2 + \cdots + \binom{n}{n-1}x^{n-1} + \binom{n}{n}x^n  \text{ , and}   
\end{equation*}
\begin{equation*}
(x + 1)^n = \binom{n}{0}x^n + \binom{n}{1}x^{n-1} + \binom{n}{2}x^{n-2} + \cdots + \binom{n}{n-1}x^1 + \binom{n}{n}x^0   
\end{equation*}
If we multiply these two equations, but only look at the coefficients of $x^n$ we obtain : 
\begin{equation*}
\binom{2n}{n} x^n = x^n\left(\binom{n}{0}^2 + \binom{n}{1}^2 + \binom{n}{2}^2 + \cdots + \binom{n}{n-1}^2 + \binom{n}{n}^2 \right) 
\end{equation*}

and the result follows immediately.
%\clearpage  %forces a new page. 

\clearpage
\subsection{Pascals triangle}

The above equations have an interesting 'visual' appeal when we look at Pascal's triangle.

\begin{table}[!hbp]
\begin{tabular}{llllllllllllllllll}
\textbf{n=0} &   &   &   &   &            &            &             &             & 1           &             &             &            &            &   &   &   &   \\
\textbf{n=1} &   &   &   &   &            &            &             & 1           &             & 1           &             &            &            &   &   &   &   \\
\textbf{n=2} &   &   &   &   &            &            & 1           &             & 2           &             & 1           &            &            &   &   &   &   \\
\textbf{n=3} &   &   &   &   &            & 1          &             & 3           &             & 3           &             & 1          &            &   &   &   &   \\
\textbf{n=4} &   &   &   &   & \textcolor{red}{\textbf{1}} &            & \textcolor{red}{\textbf{4}}  &             & \textcolor{red}{\textbf{6}}  &             & \textcolor{red}{\textbf{4}}  &            & \textcolor{red}{\textbf{1}} &   &   &   &   \\
\textbf{n=5} &   &   &   & 1 &            & \textcolor{red}{\textbf{5}} &             & \textcolor{red}{\textbf{10}} &             & \textcolor{red}{\textbf{10}} &             & \textcolor{red}{\textbf{5}} &            & 1 &   &   &   \\
\textbf{n=6} &   &   & 1 &   & 6          &            & \textcolor{red}{\textbf{15}} &             & \textcolor{red}{\textbf{20}} &             & \textcolor{red}{\textbf{15}} &            & 6          &   & 1 &   &   \\
\textbf{n=7} &   & 1 &   & 7 &            & 21         &             & \textcolor{red}{\textbf{35}} &             & \textcolor{red}{\textbf{35}} &             & 21         &            & 7 &   & 1 &   \\
\textbf{n=8} & 1 &   & 8 &   & 28         &            & 56          &             & \textcolor{red}{\textbf{70}} &             & 56          &            & 28         &   & 8 &   & 1
\end{tabular}
\end{table}

Looking at the inverted triangle of bold red values starting at the line $n = 4$ and reaching the value $70$ at the lowest point (where $n=8$), we now know that \\
$1^2 + 4^2 + 6^2 + 4^2 + 1^2 = 70$ \\
This works for any such inverted triangle.  

\subsection{Large $n$}
We would certainly expect the value of $\binom{2n}{n} x^n$ to approach $0$ as $n \rightarrow \infty,$ since the chance of the same number of heads for an enormous number of throws seems remote. \\
The value also seem to decrease ``quite slowly'', e.g. the values for $n=8$ and $n=32$ are around $\frac{1}{5}$ and $\frac{1}{10}$ respectively. \\
Using Stirling's formula :     
\begin{equation*}
n!  \sim \sqrt{2 \pi n}(\frac{n}{e})^n \implies \binom{2n}{n}\left(\frac{1}{2}\right)^{2n} \sim \left(\frac{\sqrt{2 \pi 2n}(\frac{2n}{e})^{2n}}{\sqrt{2 \pi n}(\frac{n}{e})^n \times \sqrt{2 \pi n}(\frac{n}{e})^n}\right)\left(\frac{1}{2}\right)^{2n}
\end{equation*}
\begin{equation*}
= \frac{1}{\sqrt{\pi n}}
\end{equation*}

This is a very good approximation for the actual values, even for small $n$.

\subsection{What about an unfair coin?}
 
We cannot use the sneaky trick from Answer 1, so we can perhaps attempt a direct approach again\ldots
We now use a biased coin which shows heads with probability $p = 1 - q$ whenever it is tossed. \\

Now,
\begin{equation*}
\mathbf{P}(X = Y) = \sum_k \mathbf{P}(X = k)\mathbf{P}(Y = k) = 
 \sum_k \left(\binom{n}{k}\right)^2 p^2k q^2{n-k} 
\end{equation*}

If we try a similar trick to evaluate this, we would start with : 
\begin{equation*}
(1 +p^2 x)^n = \binom{n}{0}x^0 + \binom{n}{1}p^2x^1 + \binom{n}{2}p^2x^2 + \cdots + \binom{n}{n-1}p^{2n-2}x^{n-1} + \binom{n}{n}p^{2n}x^n  \text{ , and}   
\end{equation*}
\begin{equation*}
(q^2x + 1)^n = \binom{n}{0}q^{2n}x^n + \binom{n}{1}q^{2n-2}x^{n-1} + \binom{n}{2}q^{2n-4}x^{n-2} + \cdots + \binom{n}{n-1}q^2x^1 + \binom{n}{n}x^0   
\end{equation*}

and we could try to determine the coefficient of $x^n$ in $(1+p^2x)^n (1+q^2x)^n$ \\
Unfortunately, there appears to be no naturally simple answer to this (except for the sum we are trying to determine in the first place).

I suspect there is no closed formula for the probability for the unfair coin, but I'm very happy to be corrected by any readers! 


\clearpage

\section{Appendix - A quick guide to solving linear recurrence relations}
\subsection{Introduction}

This section only deals with techniques for solving recurrence relations (or ``difference equations'') where we have a sequence of numbers 
$x_0, x_1, \ldots$ satisfying the relationship
\begin{equation} 
a_0x_{n+k} + a_1x_{n+k-1} + \cdots + a_kx_n = g(n) \qquad \text{ for } n = 0, 1, 2 \ldots
\tag{R1} \label{Recurrence1}
\end{equation}
where $a_0, a_1, \ldots a_k$ are given real numbers and both $a_0 \neq 0$ and $a_k \neq 0$, and $g(n)$ is some polynomial function of $n$ (or a constant).
\\
A famous example which can be tackled is the Fibonacci sequence for which the relationship is $x_{n+2} = x_{n+1} + x_n$ (or equivalently  $x_{n+2} - x_{n+1} - x_n = 0$) with $x_0=0$ and $x_1=1$.
Example 2 below covers this. 
\\
To get an explicit solution, we also assume that we know the initial values $x_0, x_1, \ldots x_{k-1}$ (although any $k$ known values would suffice). 
\\
The first step is to find the roots of the \emph{auxiliary equation} (sometimes called the 'characteristic equation') :  
\begin{equation} 
a_0\theta^{k} + a_1\theta^{k-1} + \cdots + a_{k-1}\theta + a_k = 0
\end{equation}

If all of the roots (including complex roots) $\theta_1, \theta_2, \ldots \theta_k$ are distinct, then the general solution of  \eqref{Recurrence1} (where $g(n) \equiv 0$ for now) is :
\begin{equation} 
x_n = c_1\theta_1^{n} + c_2\theta_2^{n} + \cdots + c_k\theta_k^{n} \qquad \text{ for } n = 0, 1, 2 \ldots
\end{equation}
and $c_1, c_2, \ldots c_k$ are arbitrary constants which can be found by equating the first $k$ values with the known initial values, and solving the simultaneous equations.
  
If any roots are the same (and occur $m$ times), then the term for that root $\theta$ will instead be of the form 
$(s_1 + s_2n + \cdots + s_mn^{m-1})\theta^n$

Finally, for $g(n) \not\equiv 0$  we firstly solve as above for $g(n) \equiv 0$ to obtain the \emph{complementary solution} $x_n = \kappa_n$, then by any devious/clever means required, 
we find a \emph{particular solution} for our particular $g(n)$, $x_n = \pi_n$
\\
The general solution is then simply $x_n = \kappa_n + \pi_n$.

\subsection{Example 1}
Solve $x_{n+2} - 7x_{n+1} + 10x_n = 0$ where $x_0=2$ and $x_1=3$. \\
The auxiliary equation is $\theta^2 - 7 \theta + 10 = 0 \implies \theta_1 = 2, \theta_2 = 5$ (the root order is not important). \\
The general solution is therefore of the form :
\begin{equation} 
x_n = c_1 2^n + c_2 5^n
\end{equation}
\\
Setting $n=0$ and $n=1$ respectively, we have :
\begin{IEEEeqnarray*}{l}
2 = c_1 + c_2 \\
3 = 2c_1 + 5c_2 \\
\end{IEEEeqnarray*}
which leads to $c_1 = \frac{7}{3}$ and $c_2 = \frac{-1}{3}$ \\
The general solution is therefore :  
\begin{equation} 
x_n = \frac{7}{3} 2^n - \frac{1}{3} 5^n
\end{equation}

\subsection{Example 2 The Fibonacci sequence}
Here, we solve $x_{n+2} - x_{n+1} - x_n = 0$ with $x_0=0$ and $x_1=1$. \\

The auxiliary equation is $\theta^2 - \theta - 1 = 0 \implies \theta_1 = \frac{1+\sqrt5}{2}, \theta_2 = \frac{1-\sqrt5}{2}$ (again the root order is not important). \\
$\theta_1$ is the ``golden ratio'' $1.6180339\ldots$ and $\theta_2$ is $-0.6180339\ldots$.  
\\
The general solution is therefore of the form :
\begin{equation} 
x_n = c_1 \left(\frac{1+\sqrt 5 }{2}\right)^n + c_2 \left(\frac{1-\sqrt 5 }{2}\right)^n
\end{equation}
\\
Setting $n=0$ and $n=1$ respectively, we have : 
\begin{IEEEeqnarray*}{l}
0 = c_1 + c_2 \\
1 = \left(\frac{1+\sqrt 5 }{2}\right)c_1 + \left(\frac{1-\sqrt 5 }{2}\right)c_2 \\
\end{IEEEeqnarray*}
which leads to $c_1 = \frac{1}{\sqrt 5 }$ and $c_2 = -\frac{1}{\sqrt 5 }$ \\

So, the general solution is therefore :  
\begin{equation} 
x_n = \frac{ \left( \frac{1+\sqrt 5 }{2}\right)^n -  \left( \frac{1-\sqrt 5 }{2}\right)^n   }{\sqrt 5 }
\end{equation}

This formula naturally explains why the ratio of successive Fibonacci numbers approaches the golden ratio as $n$ increases.

\subsection{Example 3}
Solve $x_{n+3} - 5x_{n+2} + 8x_{n+1} - 4x_n = 0$ where $x_0=0$ , $x_1=3$, $x_2=13$ \\
The auxiliary equation is $\theta^3 - 5 \theta^2 + 8 \theta - 4 = 0 \implies \theta_1 = 2, \theta_2 = 2, \theta_3 = 2$ (again the root order is not important). \\
The general solution is therefore of the form :
\begin{equation} 
x_n = c_1 1^n + (c_2 + c_3n) 2^n
\end{equation}
\\
The boundary conditions can then be used to determine $c_1 = 1$ , $c_2 = -1$ and $c_3 = 2$

\subsection{Example 4}
Solve $x_{n+2} - 5x_{n+1} + 6x_n = 4n+2$ where $x_0=5$ , $x_4=-37$ \\
As above, the complementary solution is of the form 
\begin{equation} 
x_n = c_1 2^n + c_2 3^n
\end{equation}
\\
By trying a general polynomial of the same degree as $4n+2$ (i.e. $sn+t$) which will always work provided there are no repeated roots of the auxiliary equation), we then find that $x_n = 2n+4 $ for $n=0, 1, 2, \ldots$ is a particular solution, so the general solution is therefore   
\begin{equation} 
x_n = c_1 2^n + c_2 3^n + 2n + 4
\end{equation}
The boundary conditions can then be used to determine $c_1 = 2$ and $c_2 = -1$.
\\  
\qquad \\ 
\qquad \\ 
\section{Acknowledgements}
I'm grateful to Anthony Robin for spotting a couple of typos in my original submission and the suggestion regarding including the Fibonacci sequence as an example.
\end{flushleft}
\end{document}
