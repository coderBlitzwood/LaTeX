\documentclass[a4paper,10pt]{article}
\pagestyle{plain}
%\pdfpagewidth=\paperwidth
%\pdfpageheight=\paperheight
\addtolength{\textheight}{3.1cm}
\addtolength{\textwidth}{1.1cm}
\addtolength{\voffset}{-2.1cm}
%\addtolength{\textheight}{1.1cm}
\usepackage{amsmath}
\usepackage{amssymb}
\usepackage{mathrsfs}
\author{Andy Mitchell}
% Define the title
\title{Formulae from ``Probability - an Introduction'' by a couple of people or more   s(1985)} 
\begin{document} 
\maketitle 
\begin{flushleft}
\section{Chapter 1}
He also mentions a couple of examples in which the maximum number of unit fractions are required :-

\begin{equation} 
\frac{4}{25} = \frac{1}{7} + \frac{1}{59} +  \frac{1}{5163} +  \frac{1}{53307975}
\end{equation}

\begin{equation} 
\mathscr{ABCDEFGHIJKLMNOPQRSTUVWXYZ}
\end{equation}



\begin{equation*} 
\frac{5}{241} = \frac{1}{49} + \frac{1}{2953} +  \frac{1}{11623993} +  \frac{1}{202675808272081} +  \frac{1}{82154966517482471538039869041} 
\end{equation*}
\section{Chapter 2}
DEFEEE
\end{flushleft}
\end{document}
